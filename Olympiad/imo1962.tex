\documentclass{article}
\usepackage{gvv-book}
\usepackage{gvv}
\begin{document}
\begin{enumerate}
\item Find the smallest natural number n which has the following properties:
\begin{enumerate}[label=(\alph*)]
\item Its decimal representation has $6$ as the last digit.
\item If the last digit $6$ is erased and placed in front of he remaining digits, the resulting number is four times as large as the original number n.
\end{enumerate}
\item Determine all real numbers $x$ which satisfy the inequality: \begin{align}
\sqrt{3-x}-\sqrt{x+1}>\frac{1}{2} \end{align}
\item consider the cube $ABCDA'B'C'D'(ABCD and A'B'C'D'$ are the upper and lower bases, respectively,and edges $AA',BB',CC',DD'$ are parallel).The point $X$ moves at constat speed along the perimeter of the square $ABCD$ in the direction $ABCDA$,and the point $Y$ moves at the same rate along the perimeter of the square $B'C'C B$ in the direction $B'C'C B B'$. points $X$ and $Y$ begin their motion at the same instant from the starting positions $A and B'$, respectively. Determine and draw the locus of the midpoints of the segments $XY$.
\item solve the equation $\cos^{2}x+\cos^{2}2x+\cos^{2}3x=1$
\item On the circle $K$ there are given three distinct points $A, B, C$. Construct (using only straightedge and compasses) a fourth point $D$ on $K$ such that a circle can be inscribed in the quadrilateral thus obtained.
\item Consider an isosceles triangle. Let r be the radius of its circumscribed circle and $\rho$ the radius of its inscribed circle. Prove that the distance $d$ between the centers of these two circles is \begin{align*}
$d=\sqrt{r(r-2\rho)}$ \end{align*}
\item The tetrahedron $SABC$ has the following property: there exist spheres, each tangent to the edges $SA, SB, SC, BCCA, AB,$ or to their extensions.
\begin{enumerate}[label=(\alph*)]
	\item Prove that the tetrahedron $SABC$ is regular.
	\item Prove conversely that for every regular tetrahedron five such spheres exist.
\end{enumerate}
\end{enumerate}    
\end{document}
